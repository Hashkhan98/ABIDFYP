\documentclass{UoNMCHA}
\usepackage[authoryear]{natbib}
\usepackage{array,booktabs} % For nice tables
\usepackage{amsmath,amsfonts,amssymb} % For nice maths
\usepackage{color}
\usepackage{enumerate}
\usepackage{listings}
\usepackage{subfig}
\usepackage{hyperref}
\usepackage[parfill]{parskip}   % For replacing paragraph indenting with a newline instead

% Number equations per section
\numberwithin{equation}{section}

\hypersetup{
%    bookmarks=true,         % show bookmarks bar?
%    unicode=false,          % non-Latin characters in AcrobatÕs bookmarks
%    pdftoolbar=true,        % show AcrobatÕs toolbar?
%    pdfmenubar=true,        % show AcrobatÕs menu?
%    pdffitwindow=false,     % window fit to page when opened
%    pdfstartview={FitH},    % fits the width of the page to the window
%    pdftitle={My title},    % title
%    pdfauthor={Author},     % author
%    pdfsubject={Subject},   % subject of the document
%    pdfcreator={Creator},   % creator of the document
%    pdfproducer={Producer}, % producer of the document
%    pdfkeywords={keyword1} {key2} {key3}, % list of keywords
%    pdfnewwindow=true,      % links in new window
    colorlinks=true,       % false: boxed links; true: colored links
    linkcolor=blue,          % color of internal links
    citecolor=blue,        % color of links to bibliography
%    filecolor=magenta,      % color of file links
    urlcolor=blue           % color of external links
}

\definecolor{MATLABKeyword}{rgb}{0,0,1}
\definecolor{MATLABComment}{rgb}{0.1328125,0.54296875,0.1328125}
\definecolor{MATLABString}{rgb}{0.625,0.125,0.9375}

\lstset{language=Matlab,
    basicstyle=\small\ttfamily,
    keywordstyle=\color{MATLABKeyword},
    %identifierstyle=,
    commentstyle=\color{MATLABComment},
    stringstyle=\color{MATLABString},
    numberstyle=\tiny,
    %numbers=left,
    basewidth=0.5em}

\firstpage{1}    % Set page number for first page
\UoNMCHAreportNo{Final Year Project Report} %Report number
\UoNMCHAyear{2019}   % Year
\shorttitle{Design of a Beam-Type 3D Printer} %For odd pages
%%%%%%%%%%%%%%%%%%%%%%%%%%%%%%%%%%%%%%%%%%%%%%%%%%%%
\begin{document}
\title{Beam type 3D Printer \\ \ \\
{\small Final Year Project  \\November 2019}}
\author[UoNMCHA]{Abid Khan}
\address[UoNMCHA]{
Student of Mechatronics Engineering,\\
The University of Newcastle, Callaghan, NSW 2308, AUSTRALIA \\
Student Number: 3189548 \\
E-mail: \href{mailtoc3189548@uon.edu.au}{\textsf{c3189548@uon.edu.au}}}
%%%%%%%%%%%%%%%%%%%%%%%%%%%%%%%%%%%
\maketitle
\onecolumn

\vspace{-5mm}
\section*{Executive summary}
%\vspace{-3mm}
In this report, is used the Lagrangian classical mechanics for modelling the dynamics of a fully Aactuated system, specifically a beam type 3D printer. The system consists of one rotational and three prismatic joints which have 4 equations of motion. A draft model is created in Creo Parametric which based on the material and dimensions of the model provides some physical variables necessary for modelling with the intention to be imported into SimMechanics of MATLAB with the mathematical model which would be consisting of Euler-Lagrange’s equations implemented in Simulink MATLAB, solved with the ODE23tb method, included in the MATLAB libraries for the solution of systems of equations of the type and order obtained. This report will also cover the design and hardware approach for the project.

%\begin{itemize}
%    \item Defines the intention of the report.
%    \item Places the report in context so the reader knows why it is important to read it.
%    \item Why is it important?
%    \item What problem is addressed?
%    \item Briefly states the results
%    \item Briefly presents the implications and recommendations
%\end{itemize}
%Executive summaries can take from a couple of paragraphs to a couple of pages.

\vspace{-2mm}
\section*{Acknowledgements}
\vspace{-3mm}
Dr.Alejandro Donaire has provided significant amount of help and support with this project. He is also the supervisor for this project.
\newpage
\tableofcontents
\listoffigures
\newpage
%%%%%%%%%%%%%%%%%%%%%%%%%%%%%%%
\section{Introduction}
The project aims at designing and building a proof-of-concept prototype of a 3D printer for house constructions. The first stage of the project focuses on the modelling, analysis, simulation and control and evaluation of a rotational machine with prismatic arm, the 3D printer, that moves in its workspace where the walls of the house are to be printed. An active counterweight balancing system attached to the printer to increase the precision and speed of the machine. In a second stage of the project, a scaled protype of the 3D printer will be built to demonstrate the concept.
\begin{itemize}
    \item \textbf{Position}: Show there is a problem and that it is important to solve it.
    \item \textbf{Problem}: Describe the specifics of the problem you are trying to address
    \item \textbf{Proposal}: Discuss how you are going to address this problem. Use the literature to back-up your approach to the problem, or to highlight that what you are doing has not been done before
\end{itemize}
Here you need to sell why what you are doing is important, and what benefits will it bring if you are successful and solve the problem? 
%

The rest of the report is organised as follows. Section~\ref{sec:Core Section} describes items related to the core content. Section~\ref{sec:Conclusion} concludes the report. Appendix~\ref{app:Table} shows an example of how to make a Table.
%%%%%%%%%%%%%%%%%%%%%%%%%%%%%%%
\newpage
\section{Prototype proposed}
\begin{table}[h!]
	\begin{center}\label{tab:Variable System}
		\caption{Variable Sytem}\label{tab:notation}
		{\footnotesize
			\begin{tabular}{c l l l|}
				\hline
				Physical Characteristics & {Symbology} & {Units} \\ \hline 
				 Acceleration of gravity & {$\boldmath{g}$} & $\boldmath{m/s^2}$ \\
				 Angular position of the arm & Introduction to Engineering Practice\\
				 Angular position of the pendulum & Mathematics I\\
				 Generalized coordinates & Integrated Physics\\
				 Generalized velocities & Electrical Engineering I\\
				 Kinetic Energy & \textbf{Procedural Programming}  \\
				 Lagrangian & Introduction to Engineering Mechanics\\
				 Potential Energy & Mathematics II 
				\\ \hline
	
			\end{tabular}
		}
	\end{center}
\end{table}
\newpage
\section{Modelling}\label{sec:Core Section}



\LaTeX \ is very good for writing Mathematics. You can write mathematics in the middle of a sentence, like for example $y=m x + h$. Or you can use the \verb|equation| environment as indicated in \eqref{eq:EquationLine} below.
\begin{equation}\label{eq:EquationLine}
    y=m x + h.
\end{equation}
You can also use equations and tell LaTeX not to number an equation:
\begin{equation*}
    z=m_z x^2 + h_z.
\end{equation*}
You can use the split command as in \eqref{eq:SS1} below (split gives you only one equation number):
\begin{equation}\label{eq:SS1}
    \begin{split}
        \dot{\mathbf{x}} &= \mathbf{A} \mathbf{x} + \mathbf{B} \mathbf{u}, \\
        \mathbf{y} &= \mathbf{C} \mathbf{x} + \mathbf{D} \mathbf{u},
    \end{split}
\end{equation}
and you also use numbers for each equation and refer to them separately like in \eqref{eq:State} and \eqref{eq:Output} below:
\begin{align}
    \dot{\mathbf{x}} &= \mathbf{A} \mathbf{x} + \mathbf{B} \mathbf{u},  \label{eq:State} \\
    \mathbf{y} &= \mathbf{C} \mathbf{x} + \mathbf{D} \mathbf{u}. \label{eq:Output} 
\end{align}
You can write a matrix like
\begin{equation*}
    \mathbf{A} =
    \begin{bmatrix}
        A_{11} & A_{12} & \dots & &A_{1n} \\
        A_{21} & A_{22} &  \dots & &\vdots \\
        \vdots & \vdots & \ddots& &  \vdots\\
        A_{m1} & A_{m2} & \dots & &A_{mn}
    \end{bmatrix}.
\end{equation*}
If you want to distinguish vectors from scalars you can use \textbf{bold} for vectors and matrices:
\begin{equation*} 
    \begin{split}
        \dot{\mathbf{x}} &= \mathbf{A} \mathbf{x} + \mathbf{B} u, \\
        y &= \mathbf{C} \mathbf{x} + \mathbf{D} u,
    \end{split}
\end{equation*}
where $u$ and $y$ are scalar variables and $\mathbf{x}$ is a vector variable.
You can also write Greek letters in bold: $\boldsymbol{\alpha}$.
%%%%%%%%%%%%%%%%%%%%%%%%%%
\subsection{Fully Actuated System}
In the study of mechanisms are acquired two concepts very important such as the direct and indirect
action. The first consists of movement of elements by action of an actuator, while the second consists
of the action of motion transmitted by another interconnected element. Such movements are known as
DOF, so that mechanical systems or mechanisms can be classified depending on the number of DOF
and the number of actuators. The fully actuated mechanical systems are those having the same number of DOF and actuators.\\
Underactuated mechanical systems are those with fewer actuators than DOF [16].
It is important to highlight the advantages of underactuated systems, since if they do not have advantages
over fully actuated mechanical systems, it will not make sense its development. The main advantages
present in underactuated systems are energy saving and control efforts. However, these systems are
intended to perform the same functions of fully actuated systems without their disadvantages. 
%
\subsection{Euler-Lagrange}
The dynamic equations of any mechanical system can be obtained from the known Newtonian classical
mechanics, the drawback of this formalism is the use of the variables in vector form, complicating
considerably the analysis when increasing the joints or there are rotations present in the system. In these cases, it is favorable to employ the Lagrange equations, which have formalism of scale, facilitating the
analysis for any mechanical system.
In order to use Lagrange equations, it is necessary to follow four steps:
\begin{enumerate}
	
	\item Calculation of kinetic energy.
	\item Calculation of the potential energy.
	\item Calculation of the Lagrangian.
	\item Solve the equations.
\end{enumerate}

Where the kinetic energy can be both rotationally and translational, this form of energy may be a
function of both the position and the speed  $\boldmath{K(q(t),\dot{q}(t))}$.

The potential energy is due to conservative forces as the forces exerted by springs and gravity, this
energy is in terms of the position $\boldmath{U(q(t))}$.

Is defined the Lagrangian as

\begin{equation}\label{eq:Lagrangian}
\mathcal{L} = K-U
\end{equation}
    \left(\begin{array}{cccccccc} 0 & 1 & 0 & 0 & 0 & 0 & 0 & 0\\ {x_{6}}^2 & 0 & 0 & 0 & 0 & 2\,x_{1}\,x_{6} & 0 & 0\\ 0 & 0 & 0 & 1 & 0 & 0 & 0 & 0\\ 0 & 0 & {x_{6}}^2 & 0 & 0 & 2\,x_{3}\,x_{6} & 0 & 0\\ 0 & 0 & 0 & 0 & 0 & 1 & 0 & 0\\ \frac{2\,x_{1}\,\left(2\,x_{1}\,x_{2}\,x_{6}-\tau +\frac{36\,x_{3}\,x_{4}\,x_{6}}{5}\right)}{{\left({x_{1}}^2+\frac{18\,{x_{3}}^2}{5}+\frac{43931}{48000}\right)}^2}-\frac{2\,x_{2}\,x_{6}}{{x_{1}}^2+\frac{18\,{x_{3}}^2}{5}+\frac{43931}{48000}} & -\frac{2\,x_{1}\,x_{6}}{{x_{1}}^2+\frac{18\,{x_{3}}^2}{5}+\frac{43931}{48000}} & \frac{36\,x_{3}\,\left(2\,x_{1}\,x_{2}\,x_{6}-\tau +\frac{36\,x_{3}\,x_{4}\,x_{6}}{5}\right)}{5\,{\left({x_{1}}^2+\frac{18\,{x_{3}}^2}{5}+\frac{43931}{48000}\right)}^2}-\frac{36\,x_{4}\,x_{6}}{5\,\left({x_{1}}^2+\frac{18\,{x_{3}}^2}{5}+\frac{43931}{48000}\right)} & -\frac{36\,x_{3}\,x_{6}}{5\,\left({x_{1}}^2+\frac{18\,{x_{3}}^2}{5}+\frac{43931}{48000}\right)} & 0 & -\frac{2\,x_{1}\,x_{2}+\frac{36\,x_{3}\,x_{4}}{5}}{{x_{1}}^2+\frac{18\,{x_{3}}^2}{5}+\frac{43931}{48000}} & 0 & 0\\ 0 & 0 & 0 & 0 & 0 & 0 & 0 & 1\\ 0 & 0 & 0 & 0 & 0 & 0 & 0 & 0 \end{array}\right)

\\
Therefore, the Lagrangian in general terms is defined of the following way

\begin{equation}\label{eq:Lagrangian general term}
\mathcal{L}(q(t),\dot{q}(t)) = K(q(t),\dot{q}(t)) - U(q(t))
\end{equation}
\\
Finally, are defined the Euler-Lagrange equations for a system of \textit{n} DOF as follows
\\
\begin{equation}\label{eq:Lagrangian general term}
\frac{d}{dt}\left(\frac{\partial \mathcal{L}(q,\dot{q})}{\partial \dot{q_i}}\right) - \frac{\partial \mathcal{L}(q,\dot{q})}{\partial q_i} = \tau_i
\end{equation}
\\
Where $\boldmath{i}$ = 1, \dots, $\boldmath{n}$ , $\boldmath{\tau_i}$ are the forces or pairs exerted externally in each joint, besides nonconservative
forces such as friction, resistance to movement of an object within a fluid and generally those that depend
on time or speed. It will be obtained an equal number of dynamic equations and DOF.
%
\begin{figure}[ht]
    \begin{center}
        \includegraphics[width=.6\linewidth]{Figures/SincPlot}
        \caption{Here goes the caption.}
        \label{fig:Sinc}
    \end{center}
\end{figure}



\begin{figure}
	\centering
	\includegraphics[width=0.7\linewidth]{LOGO_Square}
	\caption{This is a logo}
	\label{fig:logosquare}
\end{figure}



Figure~\ref{fig:Sinc} shows a shows a plot of the function $\sin(x)/x$. 

If I need to make a simple diagram, I use powerpoint and select the drawing and save it as a pdf. For example, look at Figure~\ref{fig:MechaSys}.
\begin{figure}[ht]
    \begin{center}
        \includegraphics[width=.6\linewidth]{Figures/MechaSys}
        \caption{Here goes the caption.}
        \label{fig:MechaSys}
    \end{center}
\end{figure}
%%%%%%%
\newpage
\subsection{Lists}
To create lists use the environments \verb|itemize|, \verb|enumerate|, or \verb|description|

The following is generated using \emph{itemize}
\begin{itemize}
    \item This is item 1 
    \item This is item 2
\end{itemize}
%
The following is generated using \emph{enumerate}
\begin{enumerate}[1)]
    \item This is item 1 
    \begin{enumerate}[a)]
        \item Subitem a
        \item Subitem b
        \begin{enumerate}[i)]
            \item Subsubitem i
            \item Subsubitem ii
        \end{enumerate}
    \end{enumerate}
    \item This is item 2
\end{enumerate}
%
The following is generated using \emph{description}
\begin{description}
    \item[foo)] This is item 1 
    \item[bar)] This is item 2
\end{description}

\subsection{Code listings}

To include a syntax-highlighted code listing, you can use the \emph{listings} package. The default options are specified by the \verb|\lstset| command. There are 3 main commands, all of which can include options to override the defaults:
\begin{enumerate}
    \item \verb|\lstinline|: Command for including code fragments inline with the text, as an alternative to \verb|\verb|. For example, we might describe function prototypes such as \lstinline[language=C,breaklines=true]|int main(int argc, char *argv[])|.
    \item \verb|\begin{lstlisting}|,\ldots,\verb|\end{lstlisting}|: Environment for including a source code listing---embedded in the LaTeX source---in a box or floating environment. An example is shown in Listing~\ref{lst:sqrt}.
    \item \verb|\lstinputlisting|: Command for including a source code listing---loaded from an external file---in a box or floating environment. This method is preferred over including the code source within the LaTeX file, since the code and its documentation can always be kept in sync. An example is shown in Listing~\ref{lst:matlabserial}.
\end{enumerate}

\begin{lstlisting}[
    language=C,
    float=h,
    numbers=none,
    xleftmargin=1cm,
    frame=none,
    caption={A winning entry from the 16th International Obfuscated C Code Contest, that computes the square root of its input.\label{lst:sqrt}}
    ]
#include <stdio.h>
int l;int main(int o,char **O,
int I){char c,*D=O[1];if(o>0){
for(l=0;D[l              ];D[l
++]-=10){D   [l++]-=120;D[l]-=
110;while   (!main(0,O,l))D[l]
+=   20;   putchar((D[l]+1032)
/20   )   ;}putchar(10);}else{
c=o+     (D[I]+82)%10-(I>l/2)*
(D[I-l+I]+72)/10-9;D[I]+=I<0?0
:!(o=main(c/10,O,I-1))*((c+999
)%10-(D[I]+92)%10);}return o;}
\end{lstlisting}

\lstinputlisting[
    language=Matlab,
    float=h,
    numbers=left,
    xleftmargin=1cm,
    frame=shadowbox,
    caption={Matlab serial communication example.\label{lst:matlabserial}},
    morekeywords={try,catch}
    ]{Code/serialtest.m}


\section{References and Citations}\label{sec:RefCite}
To generate the bibliography look at the end of this document in .tex file. To make reference to the bibliography use the commands \verb|\citet{}| and \verb|\citep{}| \citep{strunk2007elements}. You can combine more than one reference in a single citation \citep{troyka1999simon, jay1995write}.



\section{Conclusion}\label{sec:Conclusion}
This is one of the most important parts of the report. In the conclusion section, you  should 
\begin{itemize}
\item briefly summarise the results,
\item reflect on the work presented, 
\item make recommendations,
\item suggest future work or improvements.
\end{itemize}

%%%%%%%%%%%%%%%%%%%%%%%%%%%%%%%%
\bibliographystyle{harvard}
\bibliography{main} % This is the .bib file where the bibliography database is stored

\appendix
\newpage
\section{Example of a Table}\label{app:Table}
 %
\begin{table}[h!]
    \begin{center}\label{tab:MCHAProg}
        \caption{Proposed Bachelor of Engineering Mechatronics Program}\label{tab:notation}
        {\footnotesize
            \begin{tabular}{c l l l|}
                \hline\hline \textbf{1st Year} & & \\
                Semester & {Course Code} & {Course Name} \\ \hline 
                1 & GENG1000 & Computer Aided Engineering \\
                1 & GENG1803 & Introduction to Engineering Practice\\
                1 & MATH1110 & Mathematics I\\
                1 & PHYS1205 & Integrated Physics\\ \hline
                2 & ELEC1300 & Electrical Engineering I\\
                2 & \textbf{GENG1003} & \textbf{Procedural Programming}  \\
                2 & GENG1001 & Introduction to Engineering Mechanics\\
                2 & MATH1120 & Mathematics II 
                \\ \hline
                %
                \hline \textbf{2nd Year} &  \\
                Semester & {Course Code} & {Course Name} \\ \hline 
                1 & ELEC1700 & Computer Engineering I\\
                1 & ELEC2700 & Computer Engineering II\\
                1 & MECH2420 & Engineering Mechanics\\
                1 & \textbf{MCHA2440} & \textbf{Computational Engineering Modelling} \\ \hline
                2 & \textbf{MCHA2000} & \textbf{Mechatronic Systems} \\
                2 & \textbf{MECH2450} & \textbf{Engineering Computations II} \\
                2 & MECH2350 & Dynamics II\\
                2 & ELEC2320 & Electrical Circuits\\ \hline
                %
                \hline \textbf{3rd Year} &  \\
                Semester & {Course Code} & {Course Name} \\ \hline 
                1 & MECH2110 & Mechanical Engineering Design I\\
                1 & ELEC4400 & Automatic Control\\
                1 & ELEC3240 & Electronics\\
                1 & ELEC3730 & Embedded Systems \\ \hline
                2 & \textbf{MECH4400} & \textbf{Computational Mechanics}  \\
                2 & MECH2700 & Thermofluids \\
                2 & \textbf{MCHA3000} & \textbf{Mechatronic System Design I} \\
                2 & \textbf{ELEC4410} & \textbf{Control System Design and Management}   \\ \hline
                % 
                \hline \textbf{4th Year} &  \\
                Semester & {Course Code} & {Course Name} \\ \hline 
                1 & \textbf{MCHA3900} & \textbf{Mechatronic System Design II} \\
                1 & PHIL3910 & Technology and Human Values\\
                1 & GENG3830& Engineering Project Management\\
                1 & FYP A & Final Year Project part A\\ \hline
                2 & GE & General Elective\\
                2 & GE & General Elective\\
                2 & FYP B & Final Year Project part B \\ \hline
            \end{tabular}
        }
    \end{center}
\end{table}
Courses in \textbf{bold} are new to the program.

\end{document}
